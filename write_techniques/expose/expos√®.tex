Exposè

Felix Morillas Navas
Matrikelnummer 1412100

RPA Tool

Prozessoptimierung ist aktuell vor allem für große Unternehmen ein wichtiges Thema. Die Aufgabenverteilung, oft über verschiedene Abteilungen hinweg, stellt die technische Kommunikation vor eine schwierige Aufgabe. Sich wiederholende Arbeitsabläufe stellen dabei eine der größten Fehlerquellen dar und es gilt diese, meist aufgrund menschlichen Fehlverhaltens verursacht, mittels autonomer Computersysteme optimiert abarbeiten zu lassen.

Problembeschreibung

Der Bestellvorgang von Einzelteilen für die Herstellung und Montage von Bussen ist kleinschrittig und aufwendig. 
Wurde vor ein paar Monaten noch händisch der Bestand der Ersatzteile geprüft und bei Bedarf eine Lieferantenbestellung mit langen Wartezeiten ausgelöst, hat man sich in der Zwischenzeit für den innovativen 3D-Druck entschieden.

Dieses Verfahren spart zwar Zeit und Geld, strapaziert aber die Arbeitsabläufe zwischen Vertrieb und Betrieb hinsichtlich der Kommunikation. Oftmals haben Sachbearbeiter mit großen Mengen inkonsistener, redundanter Daten zu kämpfen und so wird aus einem Vorgang mit grundsätzlich geringer Komplexität ein flüchtigkeitsfehleranfälliger, zeitintensiver Prozess.

Ziel meiner Arbeit war das Entwickeln eines technischen Prototyps, der die intelligente Vollautomatisierbarkeit aller notwendigen Schritte im Prüfprozess für den 3D-Druck zeigen sollte. 

Nach dem detaillierten Vergleich verschiedener frei verfügbarer RPA-Tools und der Konsequenz, dass keines der Tools die Anforderungen der Aufgabe erfüllt, habe ich mittels PyAutoGui und diversen situations-spezifischen Python-Bibliotheken wie Keras für das Maschinelle Lernen und Tesseract für die Bildverarbeitung eine neue Lösung entwickelt. 
Neben der technischen Machbarkeit war es mir außerdem wichtig, in enger Zusammenarbeit mit dem Kunden die Anforderungen hinsichtlich der Bedienbarkeit und des konkreten Nutzens meiner Software zu diskutieren und eine saubere Infrastruktur, beispielsweise mit Git und robusten Coding-Guidelines, aufzusetzen.

